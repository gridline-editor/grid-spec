\chapter{Syntax}

This chapter discusses the data representation and structure of a Grid file on 
an object and file scope.

\section{Character Set}

Grid files, on an atomic level, comprise a sequence of 8-bit bytes. Its 
character set is a subset of the American Standard Code for Information 
Interchange (ASCII). There are three classes of characters: \textit{text}, 
\textit{delimiters}, and \textit{whitespace}. These classes apply to all 
characters in the character set except those within higher-level entities; in 
such cases, separate rules apply.

\textit{Whitespace} characters (see Table \ref{tab:charset_ws}) separate 
contiguous \textit{tokens} that would otherwise combine to form a single token. 
Some Whitespace characters, \textit{newline characters} may additionally serve 
as markers denoting the end of a line.

\begin{table}[ht]
    \centering
    \caption{Whitespace characters with their respective representations.}
    \label{tab:charset_ws}
    \begin{tabular*}{.8\linewidth}{
        l@{\extracolsep{\fill}}
        l@{\extracolsep{\fill}}
        l@{\extracolsep{\fill}}
        l}
        Dec & Hex & Symbol & Name \\
        \hline
        09 & 09 & HT & Horizontal Tab \\
        10 & 0A & LF & Line Feed \\
        13 & 0D & CR & Carriage Return \\
        32 & 20 & SP & Space
    \end{tabular*}
\end{table}

When a Carriage Return character (CR) immediately precedes a Line Feed 
character (LF), it will be recognized as a single \textit{newline character}, 
which indicates an End of Line (EOL) marker. By themselves, the CR and LF 
characters are also \textit{newline characters}. However, they do not, by 
themselves, constitute an EOL marker.

\textit{Delimiter} characters (see Table \ref{tab:charset_dlm}) denote the 
start and end of a higher-level entity such as a string or formula. Like 
whitespace characters, delimiter characters separate sequences of characters 
that will otherwise combine to form a token. Delimiters should be balanced such 
that every open delimiter has a corresponding closing delimiter of the same 
type. These characters are not part of the entities they terminate or 
instantiate.

\begin{table}[ht]
    \centering
    \caption{Delimiter characters with their respective representations.}
    \label{tab:charset_dlm}
    \begin{tabular*}{.8\linewidth}{
        l@{\extracolsep{\fill}}
        l@{\extracolsep{\fill}}
        l@{\extracolsep{\fill}}
        l}
        Dec & Hex & Symbol & Name \\
        \hline
        37 & 25 & \% & Percent Sign \\
        40 & 28 & ( & Open Parenthesis \\
        41 & 29 & ) & Close Parenthesis \\
        60 & 3C & < & Less Than \\
        62 & 3E & > & Greater Than \\
    \end{tabular*}
\end{table}

\textit{Text} characters are all printable characters except whitespace and 
delimiters. A sequence of contiguous text characters forms a single token.
