\chapter{Syntax}

This chapter discusses the data representation and structure of a Grid file on an object and file scope.

\section{Character Set}

Grid files, on an atomic level, comprise a sequence of 8-bit bytes. Its character set is a subset of the American Standard Code for Information Interchange (ASCII). There are three classes of characters: \preintroref{text}{}, \preintroref{delimiter}{s}, and \preintroref{whitespace}{}. These classes apply to all characters in the character set except those within higher-level entities (see Section \ref{sec:tokens}); in such cases, separate rules apply.

\introref{Whitespace}{} characters (see Table \ref{tab:charset_ws}) separate contiguous \preintroref{lexeme}{s} that would otherwise combine to form a single \preintroref{token}{}. Some \postintroref{whitespace}{} characters, may additionally serve as markers denoting the end of a line. Such characters are called \introref{newline}{} characters.

\begin{table}[ht]
    \centering
    \caption{Whitespace characters with their respective representations.}
    \label{tab:charset_ws}
    \begin{tabular*}{.8\linewidth}{
        l@{\extracolsep{\fill}}
        l@{\extracolsep{\fill}}
        l@{\extracolsep{\fill}}
        l}
        Dec & Hex & Symbol & Name \\
        \hline
        09 & 09 & HT & Horizontal Tab \\
        10 & 0A & LF & Line Feed \\
        13 & 0D & CR & Carriage Return \\
        32 & 20 & SP & Space
    \end{tabular*}
\end{table}

When a Carriage Return character (CR) immediately precedes a Line Feed character (LF), it will be recognized as a single \postintroref{newline}{} character, which indicates an End of Line (EOL) marker. By themselves, the CR and LF characters are also \postintroref{newline}{} characters. However, they do not, by themselves, constitute an EOL marker.

\introref{Delimiter}{} characters (see Table \ref{tab:charset_dlm}) denote the start and end of a higher-level entity such as a string or formula. Like \postintroref{whitespace}{} characters, \postintroref{delimiter}{} characters separate sequences of characters that will otherwise combine to form a \preintroref{token}{}. \postintroref{Delimiter}{s} should be balanced such that every open \postintroref{delimiter}{} has a corresponding closing \postintroref{delimiter}{} of the same type. These characters are not part of the entities they terminate or instantiate.

\begin{table}[ht]
    \centering
    \caption{Delimiter characters with their respective representations.}
    \label{tab:charset_dlm}
    \begin{tabular*}{.8\linewidth}{
        l@{\extracolsep{\fill}}
        l@{\extracolsep{\fill}}
        l@{\extracolsep{\fill}}
        l}
        Dec & Hex & Symbol & Name \\
        \hline
        37 & 25 & \% & Percent Sign \\
        40 & 28 & ( & Open Parenthesis \\
        41 & 29 & ) & Close Parenthesis \\
        60 & 3C & < & Less Than \\
        62 & 3E & > & Greater Than
    \end{tabular*}
\end{table}

\introref{Text}{} characters are all printable characters except \postintroref{whitespace}{} and \postintroref{delimiter}{s}.
