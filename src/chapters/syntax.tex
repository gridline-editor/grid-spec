\chapter{Syntax}

This chapter discusses the data representation and structure of a Grid file on 
an object and file scope.

\section{Character Set}

Grid files, on an atomic level, comprise a sequence of 8-bit bytes. Its 
character set is a subset of the American Standard Code for Information 
Interchange (ASCII). There are three classes of characters: 
\preintroref{text}{}, \preintroref{delimiter}{s}, and 
\preintroref{whitespace}{}. These classes apply to all characters in the 
character set except those within higher-level entities (see Section 
\ref{sec:tokens}); in such cases, separate rules apply.

\introref{Whitespace}{} characters (see Table \ref{tab:charset_ws}) separate 
contiguous \preintroref{lexeme}{s} that would otherwise combine to form a 
single \preintroref{token}{}. Some \postintroref{whitespace}{} characters, may 
additionally serve as markers denoting the end of a line. Such characters are 
called \introref{newline}{} characters.

\begin{table}[ht]
    \centering
    \caption{Whitespace characters with their respective representations.}
    \label{tab:charset_ws}
    \begin{tabular*}{\tablewidth}{
        l@{\extracolsep{\fill}}
        l@{\extracolsep{\fill}}
        l@{\extracolsep{\fill}}
        l}
        Dec & Hex & Symbol & Name \\
        \hline
        09 & 09 & HT & Horizontal Tab \\
        10 & 0A & LF & Line Feed \\
        13 & 0D & CR & Carriage Return \\
        32 & 20 & SP & Space
    \end{tabular*}
\end{table}

When a Carriage Return character (CR) immediately precedes a Line Feed 
character (LF), it will be recognized as a single \postintroref{newline}{} 
character, which indicates an End of Line (EOL) marker. By themselves, the CR 
and LF characters are also \postintroref{newline}{} characters. However, they 
do not, by themselves, constitute an EOL marker.

\introref{Delimiter}{} characters (see Table \ref{tab:charset_dlm}) denote the 
start and end of a higher-level entity such as a string or formula. Like 
\postintroref{whitespace}{} characters, \postintroref{delimiter}{} characters 
separate sequences of characters that will otherwise combine to form a 
\preintroref{token}{}. \postintroref{Delimiter}{s} should be balanced such that 
every open \postintroref{delimiter}{} has a corresponding closing 
\postintroref{delimiter}{} of the same type. These characters are not part of 
the entities they terminate or instantiate.

\begin{table}[ht]
    \centering
    \caption{Delimiter characters with their respective representations.}
    \label{tab:charset_dlm}
    \begin{tabular*}{\tablewidth}{
        l@{\extracolsep{\fill}}
        l@{\extracolsep{\fill}}
        l@{\extracolsep{\fill}}
        l}
        Dec & Hex & Symbol & Name \\
        \hline
        36 & 24 & \$ & Dollar Sign \\
        37 & 25 & \% & Percent Sign \\
        40 & 28 & ( & Open Parenthesis \\
        41 & 29 & ) & Close Parenthesis \\
        60 & 3C & < & Less Than \\
        62 & 3E & > & Greater Than
    \end{tabular*}
\end{table}

\introref{Text}{} characters are all printable characters except 
\postintroref{whitespace}{} and \postintroref{delimiter}{s}.

\section{Tokens}
\label{sec:tokens}

A \introref{lexeme}{} is the smallest meaningful element in a syntax. They are 
the made up of the maximum sequence of contiguous characters that satisfy the 
conditions established by the initial character. Conditions determine the 
bounds of the \postintroref{lexeme}{} and the type of \preintroref{token}{}. A 
\introref{token}{} is an comprised of a \postintroref{lexeme}{} and its 
corresponding type. A \postintroref{lexeme}{} type and \postintroref{token}{} 
type are synonymous.

A \postintroref{token}{} type can be categorized into two classes: 
\preintroref{data}{}, and \preintroref{non-data}. \postintroref{Token}{s} 
belonging to the \introref{data}{} class contain textual or numerical values, 
which can be represented by the following token types:

\begin{itemize}
    \item \preintroref{Identifier}{}
    \item \preintroref{Literal}{}
\end{itemize}

Unlike their counterparts, \introref{non-data}{} types do not hold values. 
Rather, they tether \postintroref{data}{} token types together or allow them to 
be distinguishable from eachother. Such \postintroref{token}{s} can be 
represented by the following types:

\begin{itemize}
    \item \postintroref{Delimiter}{}
\end{itemize}

\subsection{Identifiers}

An \introref{Identifier}{} is a distinctive lablel used to refer to 
\postintroref{token}{} types of the \postintroref{data}{} class. It is a 
sequence of one or more alphanumeric characters of arbitrary length. A dollar 
sign character (\$) must be the first character and it must be followed by a 
letter. While the dollar sign is not part of the \postintroref{identifier}{} 
name, itself, it does indicate that the following sequence of 
\postintroref{text}{} characters are. Uppercase and lowercase characters are 
distinct such that \lstinline[language=grid]!$x! and 
\lstinline[language=grid]!$X! are not the same. The following are examples of 
valid \postintroref{identifier}{s}:

\begin{lstlisting}[language=grid]
$n
$HelloWorld
$C2x2
\end{lstlisting}

\subsection{Integer Literals}

An \introref{integer literal}{} is a sequence of digits representing a 
numerical constant as a whole number. An \postintroref{integer literal}{} can 
be represented in decimal and hexadecimal notations. These 
representations may also have leading zeros, but they must follow their prefix 
(if any).

An \postintroref{integer literal}{} represented by a decimal number is the 
default representation. Decimal numbers do not have a corresponding prefix 
denoting its base. However, they can be prefixed with an negative sign (-) to 
denote a negative number. Decimal numbers are comprised of a sequence of ASCII 
digits (0-9). Decimal numbers may also use underscores (\_) to break the 
sequence into more readable chunks. An underscore must not immediately follow 
another underscore.The following are examples of valid decimal 
\postintroref{integer literal}{s}:

\begin{lstlisting}[language=grid]
172
-55
00010
-0025
1_000_000
-24_250
\end{lstlisting}

An \postintroref{integer literal}{} represented by a hexadecimal number is 
comprised of a \lstinline[language=grid]!0x!- prefix followed by a sequence of 
ASCII digits (0-9) and letters (A-F and a-f).  Hexadecimal numbers may be also 
use underscores (\_) to break the sequence into more readable chunks. An 
underscore must not immediately follow another underscore. Hexadecimal 
representations should not be prefixed with a negative sign.

The following are examples of valid integer literals:

\begin{lstlisting}[language=grid]
0x12a4
0xfabC
0x1aaa_0000
0x0001_fade
\end{lstlisting}
